\documentclass[12pt]{article}                       %document class declaration - article with 12pt font size
\usepackage[utf8]{inputenc}                         %input encoding - set to UTF8
\usepackage{float}                                  %for control over floating objects - such as tables and figures
\usepackage{hyperref}                               %easier URL reference
\usepackage{amsmath}                                %mathematical facilities for LATEX by AMS
\usepackage[hmargin=3cm,vmargin=6.0cm]{geometry}    %geometric properties of page - horizontal and vertical margins are set
\topmargin=-2cm                                     
\addtolength{\textheight}{7cm}                    
\usepackage{setspace}                               
\onehalfspacing                                     %mostly 1.5 spacing is preferred
\usepackage{fitch}                                  %relatively easier natural deduction 

\begin{document}

%% TEXT NEWLINE BOLD ITALIC
text                                %single enter in .TEX file does not result in a newline
asdakjsdkahsdkahs
text\\                              %'\\' starts from the next line
new line                            %double enter is a new paragraph with indentation

\textbf{bold}, \textit{italic}      %bold text and italic text

\noindent
\verb|\noindent| nullifies the indentation due to the double enter
\paragraph{A new paragraph} begins with a leading vertical space


%%  VSPACE HSPACE
\vspace{20px}                       %vertical space determined manually
\noindent
there is a \hspace{20px} gap        %horizontal space


%%  SECTIONS
\section{Section - enumerated automatically}
\subsection{Subsection}
\subsubsection{Subsubsection}
The font sizes of the headers are predetermined by \LaTeX.

\section*{Section I}
\subsection*{Subsection I.i}
\subsubsection*{Subsubsection I.i.a}
By using $*$ we ask from \verb|pdflatex| not to use automatic enumeration



%%  MATH MODE -- IN LINE
\vspace{20px}
\section*{Math Mode}
\subsection*{In-line}
\noindent
text character x vs. the variable $x$\\
\textbf{subscript}: $x_n$ $x_{ik}$, $s'_i$ , \textbf{superscript} : $x^k$,\\
\textbf{\textit{both}}: $x_i^k$\\                   %nested bold and italized is also valid
big operator with lower and upper indexes:\\
$\sum_i=1^N x_i^2$ versus $\sum_{i=1}^N x_i^2$      %be careful with the brackets
%                               ^   ^




%% MATH MODE -- EQUATIONS
\subsection*{Equations}
\[ \sum_{i=1}^N x_i^2 \]
\[ \nabla \cdot \vec{E} = \frac{\rho}{\epsilon_0} \] 
we cannot refer to this equation. To do so we start an equation environment:
\begin{equation}
    \nabla \cdot \vec{E} = \frac{\rho}{\epsilon_0}
    \label{eq:maxwell_electrical_source}
\end{equation}
We can refer to \eqref{eq:maxwell_electrical_source} by using \verb|\label| and \verb|\eqref|.\\
Sometimes, eqref cannot link the labelled sections properly, if this happens you can recompile.

% test these by yourselves by uncommenting
%\begin{equation} 
%\begin{split}
% E(x) & = \int f(x) dx \\
% & = \int x^2 dx
%\end{split}
%\label{eq1}
%\end{equation}
%
%grouping and centering equations :
%\begin{gather*} 
%2x - 5y =  8 \\ 
%3x^2 + 9y =  3a + c
%\end{gather*}
 



%%  MATH MODE -- SPACES
\vspace{20px}
\subsection*{Spacing}
\noindent
you cannot put spaces like this in math mode $x  =    y   +   4$\\
you can put spaces with special characters like:
\begin{verbatim}
\ \quad \qquad 
\end{verbatim}
example of spacing in math mode $ x \ = \quad x \qquad + \ 1$




%%  ITEMIZE ENUMERATE
\vspace{20px}
\section*{Itemize and Enumerate}
\begin{itemize}
    \item item 1
    \item item 2 
\end{itemize}

\begin{enumerate}
	\item item
	\item item 
\end{enumerate}



%%%%TABLES%%%%
\vspace{20px}
\section*{Tables}
tables: easy construction yet exhausting due to placement issues 

\begin{table}%[H]
\small
\centering
\caption{Sample table}
\label{table:example}
\begin{tabular}{|c|c|c|c|c|c|}	%% specify column number and vertical lines
\hline 							%% line draw
\textbf{title1} & \textbf{title2} & \textbf{title3} & \textbf{title4} & \textbf{title5} & \textbf{title6} \\
\hline 
\hline 
text & 0 & text & 20 & text & $a_0$\\			%% separate columns by &
text & 0 & & & & \\
text & 0 & $foo$ & 0 & $\pi(s = 1)$ & $a_2$\\
 &  & $foo$ & 0 & $\pi(s = 3)$ & $\emptyset$\\
\hline 

\end{tabular}
\end{table}

\begin{table}[H]
	\centering
    \caption{Answer to the 5th question from 2016's THE1 -- \textit{creating boxes are tiring}}
	\begin{tabular}{*6{l}}
		$1$ & & & $\exists x (P(x) \rightarrow Q(a))$ & \textit{premise} & \\ \cline{2-6}
		
		$2$ &\multicolumn{1}{|c}{} & & $\forall y P(y)$ &\textit{assumption} &\multicolumn{1}{c|}{}\\ \cline{3-5}
		 
		$3$ &\multicolumn{1}{|c}{}&\multicolumn{1}{|c}{$x_0$}& $P(x_0) \rightarrow Q(a)$ &\multicolumn{1}{l|}{\textit{assumption}}&\multicolumn{1}{c|}{}\\ 
		
		$4$ &\multicolumn{1}{|c}{}&\multicolumn{1}{|c}{}& $P(x_0)$ &\multicolumn{1}{l|}{$\forall y$e $2$}&\multicolumn{1}{c|}{}\\
		
		$5$ &\multicolumn{1}{|c}{}&\multicolumn{1}{|c}{}& $Q(a)$ & \multicolumn{1}{c|}{$\rightarrow$e $3,4$} &\multicolumn{1}{c|}{}\\ \cline{3-5}
		
		$6$ &\multicolumn{1}{|c}{}& & $Q(a)$ & $\exists$e $1,3-5$ & \multicolumn{1}{c|}{}\\ \cline{2-6}
		
		$7$ & & & $\forall y P(y) \rightarrow Q(a)$ & $\rightarrow$i $2-6$ & \\
	
	\end{tabular}
\end{table}

\begin{table}[H]
	\centering
	\caption{Things of these kind are also accepted as long as you make it clear where your assumption boxes end.}
	\begin{tabular}{*4{l}}
		$1$ & & $\exists x (P(x) \rightarrow Q(a))$ & \textit{premise} \\ \hline \hline     %double hline call for the outer
		
		$2$ & & $\forall y P(y)$ &\textit{assumption} \\ \hline                             %single hline call for the inner
		
		$3$ & $x_0$ & $P(x_0) \rightarrow Q(a)$ &\textit{assumption}\\ 
		
		$4$ & & $P(x_0)$ &$\forall y$e $2$\\
		
		$5$ & & $Q(a)$ & $\rightarrow$e $3,4$\\ \hline                                      %single hline call for the inner
		                                                                                                                     
		$6$ & & $Q(a)$ & $\exists$e $1,3-5$ \\  \hline \hline                               %double hline call for the outer
		
		$7$ & & $\forall y P(y) \rightarrow Q(a)$ & $\rightarrow$i $2-6$ \\
		
	\end{tabular}
\end{table}
\vspace{20px}
\begin{table}[H]
    \centering
	\begin{tabular}{*6{l}}
	$1$ & & & $\exists x (P(x) \rightarrow Q(a))$ & \textit{premise} & \\ \cline{2-6}
	
	$2$ & & & $\forall y P(y)$ &\textit{assumption} & \\ \cline{3-5}
	
	$3$ & & $x_0$ & $P(x_0) \rightarrow Q(a)$ &\textit{assumption}& \\ 
	
	$4$ & & & $P(x_0)$ &$\forall y$e $2$&\\
	
	$5$ & & & $Q(a)$ & $\rightarrow$e $3,4$&\\ \cline{3-5}
	
	$6$ & & & $Q(a)$ & $\exists$e $1,3-5$ & \\ \cline{2-6}
	
	$7$ & & & $\forall y P(y) \rightarrow Q(a)$ & $\rightarrow$i $2-6$ & \\
\end{tabular}
\end{table}

%Fitch environment
\vspace{20px}
\section*{Fitch.sty}
You can also use the style package \verb|fitch.sty| which is much easier to manage
\begin{equation*}
\begin{fitch}
    \fh \exists x (P(x) \rightarrow Q(a))                   & premise \\
    \fa \fh \forall y P(y)                                  & assumption \\
    \fa \fa \fitchmodal{x_0} P(x_0) \rightarrow Q(a)        & \\
    \fa \fa \fa P(x_0)                                      & $\forall$e, 2 \\
    \fa \fa \fa Q(a)                                        & $\rightarrow$e, 3,4\\
    \fa \fa Q(a)                                            & $\exists$e, 1,3-5\\
    \fa \forall y P(y) \rightarrow Q(a)                     & $\rightarrow$i, 2-6
\end{fitch}
\end{equation*}

\noindent
See \href{https://github.com/cbcafiero/fitchsty/blob/master/fitchdoc.pdf}{here} for a
    documentation of fitch version 0.5.\\
At the time fitch.sty is not available on \verb|inek|s.\\
To install it locally either compile with the file \verb|fitch.sty| under the same
    directory as your .tex file, or move it to somewhere where \verb|pdflatex| usually
    checks.\\
For \verb|fitch.sty| this would be \verb|$HOME/texmf/tex/latex/fitch/|.

\vspace{20px}
\noindent
You might instead use the style package \verb|logicproof.sty| useful which is readily
    available at \verb|inek|s.\\
See \href{https://tex.stackexchange.com/questions/268079/fitch-style-predicate-logic-proof}{this link}.

\section*{Connecting to ineks using \textit{ssh} and \textit{sftp}}
Upload your files using \verb|sftp| and compile them at an \verb|inek|.\\
Do not forget to upload \verb|fitch.sty| if you are using it.
\begin{verbatim}
@local>> cd $ASSIGNMENT_PATH
@local>> sftp -P 8085 eXXXXXXX@divan.ceng.metu.edu.tr
[yes]
(password)
@divan>> put fitch.sty
@divan>> put the1.tex
@divan>> exit
@local>> ssh eXXXXXXX@divan.ceng.metu.edu.tr -p 8085
[yes]
(password)
@divan>> ssh inek1
[yes]
(password)
@inek1>> pdflatex *.tex
@inek1>> ls
(the1.pdf should be among the listed)
\end{verbatim}

\end{document}
