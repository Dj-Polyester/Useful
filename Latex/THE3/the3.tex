\documentclass[12pt]{article}
\usepackage[utf8]{inputenc}
\usepackage{float}
\usepackage{amsmath}
\usepackage{amssymb}


\usepackage[hmargin=3cm,vmargin=6.0cm]{geometry}
%\topmargin=0cm
\topmargin=-2cm
\addtolength{\textheight}{6.5cm}
\addtolength{\textwidth}{2.0cm}
%\setlength{\leftmargin}{-5cm}
\setlength{\oddsidemargin}{0.0cm}
\setlength{\evensidemargin}{0.0cm}

\setlength{\parindent}{4em}
\setlength{\parskip}{1em}
\renewcommand{\baselinestretch}{1.5}

%misc libraries goes here



\begin{document}

\section*{Student Information } 
%Write your full name and id number between the colon and newline
%Put one empty space character after colon and before newline
Full Name : Batuhan Karaca \\
Id Number : 2310191 \\

% Write your answers below the section tags
\section*{Answer 1}

We know if $a \equiv 1\ (mod\ p)$ , $a^{2}\ mod\ p = (a\ mod\ p)(a\ mod\ p) = 1$.
Similarly, $a^{3}\ mod\ p = (a^{2}\ mod\ p)(a\ mod\ p) = 1$, $a^{4}\ mod\ p = (a^{3}\ mod\ p)(a\ mod\ p)$. 
We have: \\ \\
\begin{tabular}{l}
    $a^{n}\ mod\ p = (a^{n-1}\ mod\ p)(a\ mod\ p) = 1 \quad (n \in \mathbb{Z^{+}})$\\
    $a^{n} \equiv 1\ (mod\ p)$ \\
\end{tabular}
\\ \\
We know, $x^{y} \equiv 1\ (mod\ p)$. Substituting:
\\ \\
\begin{tabular}{l}
    $(x^{y})^{n} \equiv 1\ (mod\ p)$\\
    $x^{ny} \equiv 1\ (mod\ p)$\\
\end{tabular}
\\ \\
Assume, there exists $b \in \mathbb{Z^+}$ such that $y \nmid b$, $b>y$ and $x^{b} \equiv 1\ (mod\ p)$.
Then $b$ is of the form $n_0y+k$ such that $k,n_0 \in \mathbb{Z^+}$ and $k < y$:
\\ \\
\begin{tabular}{l}
    $x^{n_0y+k} \equiv 1\ (mod\ p)$\\
    $x^{n_0y}x^{k} \equiv 1\ (mod\ p)$\\
    $x^{n_0y}x^{k}\ mod\ p = (x^{n_0y}\ mod\ p)(x^{k}\ mod\ p) = (x^{k} mod\ p)$\\
    $x^{k} \equiv 1\ (mod\ p)$\\
\end{tabular}
\\ \\
However, since $k < y$ and $k \neq 0$, this is not the case. Hence by contradiction, 
every $s \in \mathbb{Z^+}$ satisfying $x^s \equiv 1 (mod\ p)$ is of the form $ny$.
\\ \\
Since $p$ is prime such that $p \nmid x$, by Fermat's Little Theorem:
\\ \\
\begin{tabular}{l}
    $x^{p-1} \equiv 1\ (mod\ p)$\\
\end{tabular}
\\ \\
Since $p$ is prime, $p > 1$ then $p-1 \in \mathbb{Z^+}$. Furthermore $p$ satisfies $x^{p-1} \equiv 1\ (mod\ p)$.
Then $p-1$ is in the set $S=\{s\ |\ s=ny\}$. Then for some $n=k$:
\\ \\
\begin{tabular}{l}
    $p-1=ky$\\
    $(p-1)/y=k$\\
\end{tabular}
\\ \\
Since $k \in \mathbb{Z^+}$, $y\ |\ p-1$.

\section*{Answer 2}
Assume $p(n)=2n^2+10n-7$
\\ \\
\begin{tabular}{l}
    $p(n)=2n^2+10n-7$\\
    $p(n)=2n^2+10n-72+65$\\
    $p(n)=(2n+18)(n-4)+65$\\
    $p(n)=2(n+9)(n-4)+65$\\
\end{tabular}
\\ \\
The Fundemental Theorem Of Arithmetic is defined as
\\ \\
\begin{tabular}{l}
    \textit{Every integer greater than 1
    can be written uniquely as a prime or as the product of two or }\\
    \textit{more primes, where the prime
    factors are written in order of nondecreasing size.}\\
\end{tabular}
\\ \\
In the textbook (8th Edition) page 272. Then:
\\ \\
\begin{tabular}{l}
    $p(n)=p_0^{a_0}*p_1^{a_1}*p_2^{a_2}*...*p_k^{a_k} \quad (a_i \in \mathbb{N})\ (1)$\\
\end{tabular}
\\ \\
such that $p_i$ is in the set of prime numbers and $p_i < p_{i+1}$.
Another theorem from the textbook, page 252 (8th Edition), Theorem 1:
\\ \\
\begin{tabular}{l}
    \textit{Let $a$, $b$, and $c$ be integers, where $a \neq 0$. Then}\\
    \textit{if $a\ |\ b$ and $a\ |\ c$, then $a\ |\ (b + c)$}$\quad (2)$\\
\end{tabular}
\\ \\
Assume for every $n \in \mathbb{Z^+}$ satisfying $13\ |\ p(n)$ holds. Then since $13\ |\ p(n)$ and $13\ |\ -65$,
\begin{tabular}{l}
    $13\ |\ p(n)-65$\\
    $13\ |\ 2(n+9)(n-4)$\\
    $13\ |\ (n+9)(n-4) \quad (since\ 13 \nmid 2)$
\end{tabular}
\\ \\
Since $13$ is a prime number, at least one of the set $\{(n+9),(n-4)\}$ is divisible by $13$.
Since $(n+9)=(n-4)+13$ and $(n-4)=(n+9)+(-13)$, similarly by $(2)$, if $13\ |\ (n-9)$ it also divides $13\ |\ (n-4)$ and
vice versa. Hence $169\ |\ (n+9)(n-4)$, but $169\ \nmid 2$ and $169\ \nmid 65$, we have
\\ \\
\begin{tabular}{l}
    $p(n)\equiv 65\ (mod\ 169)$\\
\end{tabular}
\\ \\
for some $t \in \mathbb{N}$. Furthermore:
\\ \\
\begin{tabular}{l}
    $p(n) \equiv 2(n+9)(n-4)\ (mod\ 13)$\\
\end{tabular}
\\ \\
Assume for every $n \in \mathbb{Z^+}$ satisfying $13 \nmid p(n)$ holds. Then using $(1)$
\\ \\
\begin{tabular}{l}
    $p(n)=p_0^{a_0}*p_1^{a_1}*p_2^{a_2}*...*13^{0}*...*p_k^{a_k} \quad (a_i \in \mathbb{N})\ (1)$\\
\end{tabular}
\\ \\
If $169\ |\ p(n)$:
\\ \\
\begin{tabular}{l}
    $p(n)=p_0^{a_0}*p_1^{a_1}*p_2^{a_2}*...*13^{2+t}*...*p_k^{a_k} \quad (a_i \in \mathbb{N})\ (1)$\\
\end{tabular}
\\ \\
for some $t \in \mathbb{N}$. However, $t+2\geq 2 > 0$. Then by contradiction, $169\nmid  p(n)$
\\ \\
We proved $169\nmid p(n)$ for every $n \in \mathbb{Z^+}$ satisfying $13\ |\ p(n)$ holds, and 
for every $n \in \mathbb{Z^+}$ satisfying $13 \nmid p(n)$ holds .
Since every $n \in \mathbb{Z^+}$ satisfying $13\ |\ p(n)$ or $13 \nmid p(n)$ comprise the set 
$\mathbb{Z^+}$, we proved for every $n \in \mathbb{Z^+}$, $169\nmid p(n)$.
\\ \\
\section*{Answer 3}
Since $a \equiv b\ (mod\ m)$, $m\ |\ a-b$, and since $a \equiv b\ (mod\ n)$, $n\ |\ a-b$.

The Fundemental Theorem Of Arithmetic is defined as
\\ \\
\begin{tabular}{l}
    \textit{Every integer greater than 1
    can be written uniquely as a prime or as the product of two or }\\
    \textit{more primes, where the prime
    factors are written in order of nondecreasing size.}\\
\end{tabular}
\\ \\
In the textbook (8th Edition) page 272. Then we can write \underline{for all} $i \in \mathbb{N}$
\\ \\
\begin{tabular}{l}
    $m=p_0^{a_0}*p_1^{a_1}*p_2^{a_2}*...*p_k^{a_k} \quad (a_i \in \mathbb{N})$ \\
    $n=p_0^{c_0}*p_1^{c_1}*p_2^{c_2}*...*p_k^{c_k} \quad (c_i \in \mathbb{N})$ \\
\end{tabular}
\\ \\
such that $p_i$ is in the set of prime numbers and $p_i < p_{i+1}$. We know
\\ \\
\begin{tabular}{l}
    $gcd(m,n)=p_0^{min(a_0,c_0)}*p_1^{min(a_1,c_1)}*p_2^{min(a_2,c_2)}*...*p_k^{min(a_k,c_k)}=1$ \\
\end{tabular}
\\ \\
Since $a_i,c_i \in \mathbb{N}$, $min(a_i,c_i) \in \mathbb{N}$. 
If $min(a_j,c_j) \neq 0$ for some $j$, $gcd(m,n) > 1$. However, $gcd(m,n) = 1$. 
Hence by contradiction, $min(a_j,c_j) = 0$. Then $a_i$ or $c_i$ is zero. We have also
\\ \\
\begin{tabular}{l}
    $lcm(m,n)=p_0^{max(a_0,c_0)}*p_1^{max(a_1,c_1)}*p_2^{max(a_2,c_2)}*...*p_k^{max(a_k,c_k)}$ \\
    $lcm(m,n)=p_0^{z_0}*p_1^{z_1}*p_2^{z_2}*...*p_k^{z_k}$ \\
\end{tabular}
\\ \\
such that $z_i$ is $a_i$ or $c_i$.
We know $m\ |\ a-b$, then $p_i^{a_i}\ |\ a-b$ and $n\ |\ a-b$, then $p_i^{c_i}\ |\ a-b$. 
Then $a-b$ has prime factor $p_i^{d_i}$ such that $p_i^{a_i}\ |\ p_i^{d_i}$ and $p_i^{c_i}\ |\ p_i^{d_i}$, 
hence $d_i \geq max(a_i,c_i)$, which is
$d_i \geq z_i$. Since we also know $a_i$ or $c_i$ is zero, $a_i+c_i=z_i$. We have
$d_i \geq a_i + c_i$. Then $p_i^{a_i + c_i}\ |\ p_i^{d_i}$, hence $p_i^{a_i + c_i}\ |\ a-b$. We know
\\ \\
\begin{tabular}{l}
    $mn=p_0^{a_0 + c_0}*p_1^{a_1 + c_1}*p_2^{a_2 + c_2}*...*p_k^{a_k + c_k}$ \\
\end{tabular}
\\ \\
Then $mn\ |\ a-b$. Hence $a\equiv b\ (mod\ mn)$. Emphasizing on the fact that $i$ represents every natural number.
\\ \\
\section*{Answer 4}
Assume $P(n,k)=\sum_{j=1}^{n} \prod_{i=0}^{k-1} (j+i)$ and $Q(n,k)=\frac{(n+k)!}{(n-1)!(k+1)}$.
\\ \\
\textbf{BASIS STEP}\\
\begin{tabular}{l}
    $P(1,k)=\sum_{j=1}^{1} \prod_{i=0}^{k-1} (j+i)=\prod_{i=0}^{k-1} (1+i)=k!$\\
    $Q(1,k)=\frac{(1+k)!}{0!(k+1)}=k!$\\
    $P(1,k)=Q(1,k) \quad (\forall k \in \mathbb{Z^+})$\\
\end{tabular}
\\ \\
\textbf{INDUCTIVE STEP}\\
Assume $P(n,k)=Q(n,k) \quad (\forall n,k \in \mathbb{Z^+})$.
\\ \\
\begin{tabular}{l}
    $P(n+1,k)=\sum_{j=1}^{n+1} \prod_{i=0}^{k-1} (j+i)=\sum_{j=1}^{n} \prod_{i=0}^{k-1} (j+i)+\frac{(n+k)!}{n!}$\\
    $P(n+1,k)=P(n,k)+\frac{(n+k)!}{n!}$\\
    $P(n+1,k)=Q(n,k)+\frac{(n+k)!}{n!}$\\
    $P(n+1,k)=\frac{(n+k)!}{(n-1)!(k+1)}+\frac{(n+k)!}{n!}$\\
    $P(n+1,k)=\frac{(n+k)!}{(n-1)!}(\frac{1}{k+1}+\frac{1}{n})$\\
    $P(n+1,k)=\frac{(n+k)!}{(n-1)!}(\frac{n+k+1}{n(k+1)})$\\
    $P(n+1,k)=\frac{(n+k+1)!}{n!(k+1)}=\frac{(n+1+k)!}{n!(k+1)}=Q(n+1,k)$\\
    $P(n+1,k)=Q(n+1,k)$\\
    $(P(n,k)=Q(n,k)) \rightarrow (P(n+1,k)=Q(n+1,k))$\\
\end{tabular}
\\ \\
Since $P(1,k)=Q(1,k)$ and $(P(n,k)=Q(n,k)) \rightarrow (P(n+1,k)=Q(n+1,k))$ hold for all $n,k \in \mathbb{Z^+}$,
by induction, $P(n,k)=Q(n,k)$ for all $n,k \in \mathbb{Z^+}$.
\\ \\
\section*{Answer 5}
\textbf{BASIS STEP}\\
$H_0=1 \leq 7^{1}=7$, $H_1=3 \leq 7^{3}=343$ and $H_2=5 \leq 7^{5}=16807$.
\\ \\
\textbf{INDUCTIVE STEP}\\
Assume for all $a \in \{1,2,3,...,k\}, H_a \leq 7^{a}$. We have:
\\ \\
\begin{tabular}{l}
    $H_{k+1}=5H_{k}+5H_{k-1}+63H_{k-2}$\\
    $H_{k} \leq 7^{k} \rightarrow 5H_{k} \leq 5(7^{k}) \quad (1)$\\
    $H_{k-1} \leq 7^{k-1} \rightarrow 5H_{k-1} \leq 5(7^{k-1}) \quad (2)$\\
    $H_{k-2} \leq 7^{k-2} \rightarrow 63H_{k-2} \leq 63(7^{k-2}) \quad (3)$\\
\end{tabular}
\\ \\
Summing right sides of $(1), (2), (3)$:
\\ \\
\begin{tabular}{l}
    $H_{k+1}=5H_{k}+5H_{k-1}+63H_{k-2} \leq 5(7^{k})+5(7^{k-1})+63(7^{k-2})$\\
    $H_{k+1} \leq 7^{k-2}(5(7^{2})+5(7)+63)$\\
    $H_{k+1} \leq 7^{k-2}(343)$\\
    $H_{k+1} \leq 7^{k-2}(7^{3})$\\
    $H_{k+1} \leq 7^{k+1}$\\
\end{tabular}
\\ \\
Completes the proof.
\end{document}