\documentclass[11pt]{article}
\usepackage[utf8]{inputenc}
\usepackage{float}
\usepackage{amsmath}
\usepackage{logicproof}
\usepackage[hmargin=3cm,vmargin=6.0cm]{geometry}
\usepackage{amsfonts}

%\topmargin=0cm
\topmargin=-2cm
\addtolength{\textheight}{6.5cm}
\addtolength{\textwidth}{2.0cm}
%\setlength{\leftmargin}{-5cm}
\setlength{\oddsidemargin}{0.0cm}
\setlength{\evensidemargin}{0.0cm}


\begin{document}

\section*{Student Information } 
%Write your full name and id number between the colon and newline
%Put one empty space character after colon and before newline
Full Name : Batuhan Karaca \\
Id Number : 2310191 \\

% Write your answers below the section tags
\section*{Answer 1}
\begin{flushleft}
    \textbf{a)} 
\end{flushleft}
%begin subheaders
\begin{flushleft}
    \textbf{i)} $D=A \cap (B \cup C)$
\end{flushleft}
\begin{flushleft}
    \textbf{ii)} $E=(A \cap B) \cup C$
\end{flushleft}
\begin{flushleft}
    \textbf{iii)} $D=(A - B) \cup (A \cap C)$
\end{flushleft}
\begin{flushleft}
    \textbf{b)} 
\end{flushleft}
%begin subheaders
\begin{flushleft}
    \textbf{i)} 
\end{flushleft}
\begin{logicproof}{2}
    (A \times B) \times C = A \times (B \times C) & premise\\
    \begin{subproof}
        \forall x (x \in ((A \times B) \times C)) & assumption\\
        \forall x (x \in \{(a,b,c)|((a \in A) \land (b \in B)) \land (c \in C) \}) & definition of the cartesian product\\
        \forall x (x \in \{(a,b,c)|(a \in A) \land ((b \in B) \land (c \in C)) \}) & associativity\\
        \forall x (x \in (A \times (B \times C))) & definition of the cartesian product
    \end{subproof}
    \forall x ((x \in ((A \times B) \times C)) \rightarrow (x \in A \times (B \times C))) & $\rightarrow\mathrm{i}$,2--5\\
    (A \times B) \times C \subseteq A \times (B \times C) & definition of the subset\\
    \begin{subproof}
        \forall x (x \in (A \times (B \times C)) & assumption\\
        \forall x (x \in \{(a,b,c)|(a \in A) \land ((b \in B) \land (c \in C)) \}) & definition of the cartesian product\\
        \forall x (x \in \{(a,b,c)|((a \in A) \land (b \in B)) \land (c \in C) \}) & associativity\\
        \forall x (x \in ((A \times B) \times C)) & definition of the cartesian product
    \end{subproof}
    \forall x ((x \in (A \times (B \times C))) \rightarrow (x \in (A \times B) \times C)) & $\rightarrow\mathrm{i}$,8--11\\
    A \times (B \times C) \subseteq (A \times B) \times C & definition of the subset\\
    (A \times B) \times C = A \times (B \times C)
\end{logicproof}

\begin{flushleft}
    \textbf{ii)} 
\end{flushleft}

\begin{logicproof}{2}
    (A \cap B) \cap C = A \cap (B \cap C) & premise\\
    \begin{subproof}
        \forall x (x \in ((A \cap B) \cap C)) & assumption\\
        \forall x (x \in \{x|((x \in A) \land (x \in B)) \land (x \in C) \}) & definition of the intersection\\
        \forall x (x \in \{x|(x \in A) \land ((x \in B) \land (x \in C)) \}) & associativity\\
        \forall x (x \in (A \cap (B \cap C))) & definition of the intersection
    \end{subproof}
    \forall x ((x \in ((A \cap B) \cap C)) \rightarrow (x \in A \cap (B \cap C))) & $\rightarrow\mathrm{i}$,2--5\\
    (A \cap B) \cap C \subseteq A \cap (B \cap C) & definition of the subset\\
    \begin{subproof}
        \forall x (x \in (A \cap (B \cap C)) & assumption\\
        \forall x (x \in \{x|(x \in A) \land ((x \in B) \land (x \in C)) \}) & definition of the intersection\\
        \forall x (x \in \{x|((x \in A) \land (x \in B)) \land (x \in C) \}) & associativity\\
        \forall x (x \in ((A \cap B) \cap C)) & definition of the intersection
    \end{subproof}
    \forall x ((x \in (A \cap (B \cap C))) \rightarrow (x \in (A \cap B) \cap C)) & $\rightarrow\mathrm{i}$,8--11\\
    A \cap (B \cap C) \subseteq (A \cap B) \cap C & definition of the subset\\
    (A \cap B) \cap C = A \cap (B \cap C)
\end{logicproof}
\begin{flushleft}
    \textbf{iii)} 
\end{flushleft}
% \begin{logicproof}{4}
%     (A \oplus B) \oplus C = A \oplus (B \oplus C) & premise\\
%     \begin{subproof}
%         1) x \in (A \oplus B) \oplus C & \\
%         \begin{subproof}
%             x \in A \oplus B & assumption\\
%             \begin{subproof}
%                 x \in A & assumption\\
%                 x \in A \oplus (B \oplus C) &
%             \end{subproof}
%             \begin{subproof}
%                 x \in B & assumption\\
%                 x \in B \oplus C & \\
%                 x \in A \oplus (B \oplus C) &
%             \end{subproof}
%             x \in A \oplus (B \oplus C) & 
%         \end{subproof}
%         \begin{subproof}
%             x \in C & assumption\\
%             x \in B \oplus C &\\
%             x \in A \oplus (B \oplus C) &
%         \end{subproof}
%         (A \oplus B) \oplus C \subseteq A \oplus (B \oplus C) &
%     \end{subproof}
%     \begin{subproof}
%         2) x \in A \oplus (B \oplus C) & \\
%         \begin{subproof}
%             x \in B \oplus C & assumption\\
%             \begin{subproof}
%                 x \in C & assumption\\
%                 x \in (A \oplus B) \oplus C &
%             \end{subproof}
%             & \\
%             \begin{subproof}
%                 x \in B & assumption\\
%                 x \in A \oplus B & \\
%                 x \in (A \oplus B) \oplus C &
%             \end{subproof}
%             x \in (A \oplus B) \oplus C &
%         \end{subproof}
%         \begin{subproof}
%             x \in A & assumption\\
%             x \in A \oplus B &\\
%             x \in (A \oplus B) \oplus C &
%         \end{subproof}
%         A \oplus (B \oplus C) \subseteq (A \oplus B) \oplus C &
%     \end{subproof}
%     A \oplus (B \oplus C) = (A \oplus B) \oplus C &
% \end{logicproof}

\begin{tabular}{|l|l|l|l|l|l|l|}
    \hline
    $A$&$B$ &$C$ & $A \oplus B$ &$B \oplus C$ &$(A \oplus B) \oplus C$ &$A \oplus (B \oplus C)$ \\
    \hline
    $1$ & $1$ & $1$ & $0$ & $0$ & $1$ & $1$ \\
    \hline
    $1$ & $1$ & $0$ & $0$ & $1$ & $0$ & $0$ \\
    \hline
    $1$ & $0$ & $1$ & $1$ & $1$ & $0$ & $0$ \\
    \hline
    $1$ & $0$ & $0$ & $1$ & $0$ & $1$ & $1$ \\
    \hline
    $0$ & $1$ & $1$ & $1$ & $0$ & $0$ & $0$ \\
    \hline
    $0$ & $1$ & $0$ & $1$ & $1$ & $1$ & $1$ \\
    \hline
    $0$ & $0$ & $1$ & $0$ & $1$ & $1$ & $1$ \\
    \hline
    $0$ & $0$ & $0$ & $0$ & $0$ & $0$ & $0$ \\
    \hline
\end{tabular}
\begin{flushleft}
    $\>\>\>\>\>$By the membership table above, $(A \oplus B) \oplus C = A \oplus (B \oplus C)$.
\end{flushleft}

\section*{Answer 2}
\begin{flushleft}
    \textbf{a)}$f(S)=\{t|\exists(s \in S) (t=f(s))\}$ defined in the book. Since $f$ is $A \rightarrow B$ and $A_0 \in A$,
\end{flushleft}
\begin{tabular}{l l}
    & $f(A_0)=\{t|\forall(s \in A_0) (t=f(s))\}$.$f^{-1}(f(A_0))  (i)$\\
\end{tabular}
\begin{flushleft}
    If $f$ is not injective:
\end{flushleft}
\begin{tabular}{l l}
    & $ \exists (x \in A_0) \exists (y \in A_0) \exists (z \in f(A_0)) ((x \neq y) \land (f(x)=f(y)=z))$\\
\end{tabular}
\begin{flushleft}
    $\>\>\>\>\>$Since $f$ is a function $A \rightarrow B$, there exists no $p,t \in A$ such that, $f(p)=r,f(t)=s \in B$ is undefined.
    Since $A_0 \in A$, if $p,t \in A_0$, then $p,t \in A$ $(argument 1)$.
    For the value $z$, there are more than one preimages since $x \neq y$ such that $f^{-1}(z)={x,y} (case 1)$.
    Since $f$ is not injective, the set $S=f(A_0)$ includes such value $z$.
    For any set $A_0$, every function $g: A_0 \rightarrow S=f(A_0)$ is surjective. 
    Then, every element $r, s$ in the set $S$ is mapped to its multiple -more than one in quantity $(case 1)$- preimages $\{p_0,p_1,...\}, \{t_0,t_1,...\}$; or a single -one in quantity, which is $x=y,then f^{-1}(z)=x=y$- preimage $p, t$ under $f^{-1}$ $(argument 2)$. 
    
    By $(argument 1)$ and $(argument 2)$, the set comprised of such elements $p_x, t_x$, is $A_0$.Then $f^{-1}(S)=A_0$
    Hence, $f{-1}(f(A_0))=A_0$. Since every set is a subset of its own, $A_0 \subseteq f(f^{-1}(A_0))$.
\end{flushleft}
\begin{flushleft}
    $\>\>\>\>\>$If $f$ is injective, since we introduced the cases for every $x, y \in A_0$ which result in $f(f^{-1}(A_0))=A_0$, case with the single preimage will also hold this equality- Because this case is the subcase.
\end{flushleft}

\begin{flushleft}
    \textbf{b)}If $f$ is not surjective:
\end{flushleft}
\begin{tabular}{l l}
    & $ \exists (x \in B_0) \exists (y \in B-B_0) \forall (z \in A) ((f(z) \neq x) \lor (f(z) \neq y))$\\
\end{tabular}
\begin{flushleft}
    $\>\>\>\>\>$(i) If the set includes such $x$, then some elements in $B_0$ can not be mapped to its preimage(s)-single and multiple, defined in part(a).
    For every element that has a single preimage $z$ or multiple preimages $\{z_0,z_1,z_2,...\}$, since there may exist some $t \in B-B_0$ that has a preimage some $z \in A-A_0$, some $z$ is a member of a set $S \subseteq A$. Then we have the set $f^{-1}(B_0)=S \subseteq A$.
    Since there are some $x \in B_0$ that $f$ maps no $z$ to, we have $f(S) \subset B_0$. We have, in this case $f(f^{-1}(B_0))=f(S) \subset B_0$\\   
\end{flushleft}
\begin{flushleft}
    $\>\>\>\>\>$(ii) If $f$ is surjective,then $B$ includes no such $x$ and $y$, then every element in $B_0$ can be mapped to its preimage(s)-single and multiple.
    For every element that has a single preimage $z$ or multiple preimages $\{z_0,z_1,z_2,...\}$, since there may exist some $t \in B-B_0$ that has a preimage some $z \in A-A_0$, some $z$ is a member of a set $S \subseteq A$. Then we have the set $f^{-1}(B_0)=S \subseteq A$.
    Since there are no $x \in B_0$ that $f$ maps no $z$ to, for every $x \in B_0$, we have a preimage in $S$, then $f(S)=B_0$. Hence we have $f(f^{-1}(B_0))=f(S)=B_0$.
    We conclude by (i) and (ii), that $f(f^{-1}(B_0)) \subseteq B_0$
\end{flushleft}
 
\section*{Answer 3}
\begin{flushleft}
    \textbf{(i) $\rightarrow$ (ii)} $\>\>\>\>\>$By definition, a set is countable either it is finite, or has the same cardinality as the set of positive integers. For an infinite non empty set $A$, if $A$ is countable, it should have the property: \\     
\end{flushleft}
\begin{tabular}{l l}
    & $|A|=|\mathbb{Z}^{+}| \equiv (|A|\geq|\mathbb{Z}^{+}|) \land (|A|\leq|\mathbb{Z}^{+}|)$\\
\end{tabular}
\begin{flushleft}
    $\>\>\>\>\>$Since $|A|\leq|\mathbb{Z}^{+}|$, there is a surjective function $f:\mathbb{Z}^{+} \rightarrow A$. 
    If A is a finite non empty set, since infinite sets have larger cardinality than of finite sets $|A|\leq|\mathbb{Z}^{+}|$ holds. There is a surjective function $f:\mathbb{Z}^{+} \rightarrow A$.\\    
\end{flushleft}
\begin{flushleft}
    \textbf{(ii) $\rightarrow$ (iii)} $\>\>\>\>\>$For a non empty set $A$, if there is a surjective function $f:\mathbb{Z}^{+} \rightarrow A$, then:\\
\end{flushleft}
\begin{tabular}{l l}
    & $(|A|\leq|\mathbb{Z}^{+}|) \equiv (|\mathbb{Z}^{+}|\geq|A|)$\\
\end{tabular}
\begin{flushleft}
    $\>\>\>\>\>$There exists an injective function $f:A \rightarrow \mathbb{Z}^{+}$.\\
\end{flushleft}
\begin{flushleft}
    \textbf{(iii) $\rightarrow$ (i)} $\>\>\>\>\>$For a non empty set $A$, if there is an injective function $f:A \rightarrow \mathbb{Z}^{+}$, then:\\
\end{flushleft}
\begin{tabular}{l l}
    & $(|A|\leq|\mathbb{Z}^{+}|) \equiv (|\mathbb{Z}^{+}|\geq|A|)$\\
\end{tabular}
\begin{flushleft}
    $\>\>\>\>\>$Since, $(|A|\leq|\mathbb{Z}^{+}|)$, if $\mathbb{Z}^{+}|$ is countable, $A$ is countable. 
    We can define a bijection $f:\mathbb{Z}^{+} \rightarrow \mathbb{Z}^{+}$ which is $f(x)=x$. Then:\\    
\end{flushleft}
\begin{tabular}{l l}
    & $|\mathbb{Z}^{+}| = |\mathbb{Z}^{+}|$ \\
\end{tabular}
\begin{flushleft}
    Since the cardinality of the set is equal to the cardinality of positive integers, the set positive integers is countable. Hence the set $A$ is countable.\\
\end{flushleft}
\section*{Answer 4}
\begin{flushleft}
    \textbf{a)}$\>\>\>\>\>$By definition, a set is countable either it is finite, or has the same cardinality as the set of positive integers. Since the set of finite binary strings is finite, it is countable.     
\end{flushleft}

\begin{flushleft}
    \textbf{b)} Assume the set of infinite binary strings is countable.
    Then we could list all the elements likewise:
\end{flushleft}
\begin{tabular}{l l l}
    & $s_1=d_{11} d_{12} d_{13}$ & $ ... $\\
    & $s_2=d_{21} d_{22} d_{23}$ & $ ... $\\
    & $.$&\\
    & $.$&\\
    & $.$&\\
\end{tabular}
\begin{flushleft}
    $\>\>\>\>\>$For any string $s_n$, the string must be in the set. We will define a string $s_n$, such that:
\end{flushleft}
\begin{tabular}{l l l l}
    & $s_n=f_{1} f_{2} f_{3}...$& & $for f_{i} \neq d_{ii}$\\
\end{tabular}
\begin{flushleft}
    $\>\>\>\>\>$Since we cannot find any matching string $s_i$ for $s_n$ in the set, by contradiction the set is uncountable. 
\end{flushleft}
\section*{Answer 5}

    \begin{flushleft}
        \textbf{a)}For all integers $n \geq k$, there exists k such that $k=1$. Then: \\ 
    \end{flushleft}
    \begin{tabular}{l l}
        & $log(n)+log(n)...+log(n) \geq log(1)+log(2)+...+log(n)$ \\
        & \\
        & $nlog(n) \geq log(n!)$ \\
        & \\
        & $|nlog(n)| \geq |log(n!)|$ \\
    \end{tabular}
    \begin{flushleft}
        $\>\>\>\>\>$We proved that: \\
    \end{flushleft}
    \begin{tabular}{l l}
        & $\exists k \exists c (\forall x > k)(|xlog(x)| \geq c|log(n!)|)$ \\
    \end{tabular}
    \begin{flushleft}
        $\>\>\>\>\>$With the values of c and k such that $c=1$ and $k=1$. Hence:\\
    \end{flushleft}
    \begin{tabular}{l l}
        & $nlog(n)=\Omega(log(n!))$ \\
    \end{tabular}

    \begin{flushleft}
        $\>\>\>\>\>$We will ignore the first half of the terms. Then: \\ 
    \end{flushleft}
    \begin{tabular}{l l}
        & $(\lceil \frac{n}{2} \rceil)(\lceil \frac{n}{2} \rceil)...(\lceil \frac{n}{2} \rceil) \leq (1)(2)(3)...(n)$ \\
        & \\
        & $\lceil \frac{n}{2} \rceil^{n-\lceil \frac{n}{2} \rceil+1} \leq n!$ \\
        & \\
        & $(\frac{n}{2})^{\frac{n}{2}} \leq \lceil \frac{n}{2} \rceil^{n-\lceil \frac{n}{2} \rceil+1}$ \\
        & \\
        & $(\frac{n}{2})^{\frac{n}{2}} \leq n!$ \\
        & \\
        & $log((\frac{n}{2})^{\frac{n}{2}}) \leq log(n!)$\\
        & \\
        & $(\frac{n}{2})log(\frac{n}{2}) \leq log(n!)$\\
        & \\
        & $(\frac{n}{2})log(\frac{1}{2}) + (\frac{n}{2})log(n) \leq log(n!)$\\
        & \\
        & $(\frac{n}{2})log(n) \leq (\frac{n}{2})log(\frac{1}{2}) + (\frac{n}{2})log(n)$\\
        & \\
        & $(\frac{n}{2})log(n) \leq log(n!)$\\
        & \\
        & $nlog(n) \leq 2log(n!)$\\
        & \\
        & $|nlog(n)| \leq 2|log(n!)|$\\
    \end{tabular}
    \begin{flushleft}
        $\>\>\>\>\>$We proved that: \\
    \end{flushleft}
    \begin{tabular}{l l}
        & $\exists k \exists c (\forall x > k)(|xlog(x)| \leq c|log(n!)|)$ \\
    \end{tabular}
    \begin{flushleft}
        $\>\>\>\>\>$With the values of c and k such that $c=2$ and $k=1$. Hence:\\
    \end{flushleft}
    \begin{tabular}{l l}
        & $nlog(n)=O(log(n!))$ \\
    \end{tabular}
    \begin{flushleft}
        $\>\>\>\>\>$Since we proved $nlog(n)=O(log(n!))$ and $nlog(n)=\Omega(log(n!))$:\\
    \end{flushleft}
    \begin{tabular}{l l}
        & $nlog(n)=\Theta(log(n!))$ \\
    \end{tabular}

\begin{flushleft}
    \textbf{b)} $\forall n ((n \in \mathbb{Z}) \rightarrow (((n+1)!-n!=n(n!)) \land (2^{n+1}-2^{n}=2^{n})))$. Because we are involved in integers, 
    growth rates can be determined as such. If we order two sides one by one:\\
\end{flushleft}
\begin{tabular}{l l l l l l l l l l l l}
    & $1$ & $\bullet$ & $2$ & $\bullet$ & $3$ & $ ... $ & $n-1$ & $\bullet$ & $n$ & $\bullet$ & $n$\\
    & $2$ & $\bullet$ & $2$ & $\bullet$ & $2$ & $ ... $ & $2$ & $\bullet$ & $2$ & &\\
\end{tabular}
\begin{flushleft}
    By the commutativity rule for product:
\end{flushleft}
\begin{tabular}{l l l l l l l l l l}
    & $n$ & $\bullet$ & $2$ & $\bullet$ & $3$ & $ ... $ & $n-1$ & $\bullet$ & $n$\\
    & $2$ & $\bullet$ & $2$ & $\bullet$ & $2$ & $ ... $ & $2$ & $\bullet$ & $2$\\
\end{tabular}
\begin{flushleft}
    Then:
\end{flushleft}
\begin{tabular}{l l}
    & $\forall (n \geq 2) (n(n!) \geq 2^{n})$ \\
\end{tabular}
\begin{flushleft}
    We conclude that for all $n \geq 2$, growth rate of $n!$ is greater than of $2^{n}$. Hence $n!$ grows faster as n goes larger. \\
\end{flushleft}



\end{document}