\documentclass[11pt]{article}
\usepackage[utf8]{inputenc}
\usepackage{float}
\usepackage{amsmath}
\usepackage{logicproof}

\usepackage[hmargin=3cm,vmargin=6.0cm]{geometry}
%\topmargin=0cm
\topmargin=-2cm
\addtolength{\textheight}{6.5cm}
\addtolength{\textwidth}{2.0cm}
%\setlength{\leftmargin}{-5cm}
\setlength{\oddsidemargin}{0.0cm}
\setlength{\evensidemargin}{0.0cm}

%misc libraries goes here
\usepackage{fitch}


\begin{document}

\section*{Student Information } 
%Write your full name and id number between the colon and newline
%Put one empty space character after colon and before newline
Full Name : Batuhan Karaca \\
Id Number : 2310191 \\

% Write your answers below the section tags
\section*{Answer 1}
\begin{flushleft}
    \textbf{a)} 
\end{flushleft}
\begin{table}[H]
    \begin{tabular}{|l|l|l|l|l|}
        \hline
        $p$ & $q$ & $(q \rightarrow \neg p)$ & $(p \leftrightarrow q)$ & $(q \rightarrow \neg p) \leftrightarrow (p \leftrightarrow q)$ \\
        \hline
        $T$ & $T$ & $F$ & $T$ & $F$ \\
        \hline
        $F$ & $F$ & $T$ & $T$ & $T$ \\
        \hline
        $T$ & $F$ & $T$ & $F$ & $F$ \\
        \hline
        $F$ & $T$ & $T$ & $F$ & $F$ \\
        \hline
    \end{tabular}
\end{table}
\begin{flushleft}
    \textbf{b)} 
\end{flushleft}
\begin{table}[H]
    \begin{tabular}{|l|l|l|l|l|l|l|l|}
        \hline
        $p$ & $q$ & $r$ & $p \lor q$ & $p \rightarrow r$ & $q \rightarrow r$ & $(p \lor q) \land (p \rightarrow r) \land (q \rightarrow r)$ & $((p \lor q) \land (p \rightarrow r) \land (q \rightarrow r) ) \rightarrow r$ \\
        \hline
        $T$ & $T$ & $T$ & $T$ & $T$ & $T$ & $T$ & $T$ \\ %1
        \hline
        $T$ & $T$ & $F$ & $T$ & $F$ & $F$ & $F$ & $T$  \\ %2
        \hline
        $T$ & $F$ & $T$ & $T$ & $T$ & $T$ & $T$ & $T$ \\ %3
        \hline
        $T$ & $F$ & $F$ & $T$ & $F$ & $T$ & $F$ & $T$ \\ %4
        \hline
        $F$ & $T$ & $T$ & $T$ & $T$ & $T$ & $T$ & $T$ \\ %5
        \hline
        $F$ & $T$ & $F$ & $T$ & $T$ & $F$ & $F$ & $T$ \\ %6
        \hline
        $F$ & $F$ & $T$ & $F$ & $T$ & $T$ & $F$ & $T$ \\ %7
        \hline
        $F$ & $F$ & $F$ & $F$ & $T$ & $T$ & $F$ & $T$ \\ %8
        \hline
    \end{tabular}
\end{table}

Inferring by the truth table above, every possible truth combination of p, q and r gives the value T (true).
Hence, we can say that the expression $((p \lor q) \land (p \rightarrow r) \land (q \rightarrow r) ) \rightarrow r$
is a tautology.
\section*{Answer 2}

$\neg p \rightarrow (q \rightarrow r) \equiv \neg(\neg p) \lor (q \rightarrow r)$ TABLE 7 \\ 
$\null\>\>\>\>\>\>\>\>\>\>\>\>\>\>\>\>\>\>\>\>\>\>\>\>\>\>\>\equiv p \lor (q \rightarrow r)$ TABLE 6, by the double negation law \\
$\null\>\>\>\>\>\>\>\>\>\>\>\>\>\>\>\>\>\>\>\>\>\>\>\>\>\>\>\equiv p \lor (\neg q \lor r)$ TABLE 7 \\
$\null\>\>\>\>\>\>\>\>\>\>\>\>\>\>\>\>\>\>\>\>\>\>\>\>\>\>\>\equiv (p \lor \neg q) \lor r$ TABLE 6, by the associative laws for disjunction \\ 
$\null\>\>\>\>\>\>\>\>\>\>\>\>\>\>\>\>\>\>\>\>\>\>\>\>\>\>\>\equiv (\neg q \lor p) \lor r$ TABLE 6, by the commutative laws for disjunction \\
$\null\>\>\>\>\>\>\>\>\>\>\>\>\>\>\>\>\>\>\>\>\>\>\>\>\>\>\>\equiv \neg q \lor (p \lor r)$ TABLE 6, by the associative laws for disjunction \\
$\null\>\>\>\>\>\>\>\>\>\>\>\>\>\>\>\>\>\>\>\>\>\>\>\>\>\>\>\equiv  q \rightarrow (p \lor r)$ TABLE 7 \\

\section*{Answer 3}

\textbf{a)} $\forall x L(x,Burak)$ \\
\textbf{b)} $\forall x L(Hazal,x)$ \\
\textbf{c)} $\forall x \exists y L(x,y)$ \\
\textbf{d)} $\forall x \exists y \neg L(x,y)$ \\
\textbf{e)} $\forall x \exists y L(y,x)$ \\
\textbf{f)} $\forall x \neg (L(x,Burak) \land L(x,Mustafa))$ \\
\textbf{g)} $\exists x \exists y ((L(Ceren,x) \land L(Ceren,y) \land (x \neq y) \land \forall z(L(Ceren,z))) \rightarrow ((z=x) \lor (z=y)))$ \\
\textbf{h)} $\exists x((\forall y(L(y,x)) \land \forall \omega \forall z(L(\omega,z))) \rightarrow (z=x))$ \\
\textbf{i)} $\forall x \neg L(x,x)$ \\
\textbf{j)} $\exists x \exists y (((x=y) \rightarrow L(x,y)) \land (((x \neq y) \land L(x,y) \land \forall z(L(x,z)))) \rightarrow ((z=x) \lor (z=y)))$

\section*{Answer 4}

\begin{logicproof}{2}
    p & premise \\
    p \rightarrow (r \rightarrow q) & premise \\
    q \rightarrow s & premise \\
    \begin{subproof}
        q \lor \neg q & assumption \\
        \begin{subproof}
            q & assumption \\
            q \lor \neg q & $\lor\mathrm{i}$,5
        \end{subproof}
        q \rightarrow (q \lor \neg q) & $\rightarrow\mathrm{i}$,5--6 \\
        \begin{subproof}
            \neg q & assumption \\
            q \lor \neg q & $\lor\mathrm{i}$,8
        \end{subproof}
        \neg q \rightarrow (q \lor \neg q) & $\rightarrow\mathrm{i}$,8--9
    \end{subproof}
    q \lor \neg q & $\lor$e,4,7,10 \\
    \begin{subproof}
        q & assumption \\
        s & $\rightarrow$e,3,12 \\
        \neg q \lor s & $\lor\mathrm{i}$,13
    \end{subproof}
    q \rightarrow (\neg q \lor s) & $\rightarrow\mathrm{i}$,12--14 \\
    \begin{subproof}
        \neg q & assumption \\
        \neg q \lor s & $\lor\mathrm{i}$,16
    \end{subproof}
    \neg q \rightarrow (\neg q \lor s) & $\rightarrow\mathrm{i}$,16--17 \\
    \neg q \lor s & $\lor$e,11,15,18 \\
    \begin{subproof}
        \neg q & assumption \\
        \begin{subproof}
            q & assumption \\
            \bot & $\neg$e,20,21
        \end{subproof}
        s & $\bot$e,22 \\
        s \lor \neg r & $\lor\mathrm{i}$,23
    \end{subproof}
    \neg q \rightarrow (s \lor \neg r) & $\rightarrow\mathrm{i}$,20--24
\end{logicproof}
    




\section*{Answer 5}

\begin{logicproof}{2}
    \forall x (p(x) \rightarrow q(x)) & premise \\
    \neg \exists z r(z) & premise \\
    \exists y p(y) \lor r(a) & premise \\
    (p(t) \rightarrow q(t)) & $\forall$e,x=t,1\\
    \begin{subproof}
        r(a) & assumption\\
        \exists z r(z) & $\exists\mathrm{i}$,5\\
        \bot & $\bot\mathrm{i}$,2,6
    \end{subproof}
    \neg r(a) & $\neg\mathrm{i}$,5,7\\
    \begin{subproof}
        r(a) & assumption\\
        \bot & $\neg$e,8,9
    \end{subproof}
    \exists y p(y) & $\bot$e,10\\
    \begin{subproof}
        p(t) & assumption\\
        q(t) & $\rightarrow$e,4,12\\
        \exists z q(z) & $\exists\mathrm{i}$,13
    \end{subproof}
    \exists z q(z) & $\exists$e,11,12--14
\end{logicproof}



\end{document}